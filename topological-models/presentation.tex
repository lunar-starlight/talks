%\documentclass[handout]{beamer}
\documentclass{beamer}

% \usetheme{boxes} % see http://www.deic.uab.es/~iblanes/beamer_gallery/ for lots of examples
\usetheme{metropolis}
\usecolortheme{rose}
% \useinnertheme{circles}
% \useoutertheme{split}
% \setbeamertemplate{blocks}[rounded][shadow=true]
\geometry{}

\setbeamertemplate{navigation symbols}{} % remove navigation symbols
\setbeamertemplate{footline}{} % remove title, too long

% %next set colors - not needed
% \setbeamercolor{title}{fg=black!70!black}
% \setbeamercolor{frametitle}{fg=blue!70!black}
% \setbeamercolor{framesubtitle}{fg=green!30!black}
% \setbeamercolor{author}{fg=red!70!black}
% \setbeamercolor{institute}{fg=green!30!black}
% \setbeamercolor{date}{fg=blue!50!black}

% \usepackage[T1]{fontenc}        % kodiranje znakov v .pdf
% \usepackage[utf8]{inputenc}     % kodiranje znakov v .tex
% \usepackage[slovene]{babel}     % nastavimo slovenščino
% \usepackage{stmaryrd}

\usepackage{fontspec}
\usepackage{unicode-math}
\usepackage{enumerate}

\usepackage{graphicx}
\usepackage{ifthen}

\newcounter{thehandwritten}
\newcommand{\insertpdfpage}[3][22.5cm]{
  \begin{tikzpicture}[remember picture,overlay]
    \node[at=(current page.north),below] {
      \includegraphics[keepaspectratio=true,width=\paperwidth,page=#3,trim={0 {\dimexpr 22.5cm-#1\relax} 0 0},clip]{#2}
    };
  \end{tikzpicture}
}
\newenvironment{handwritten}[1][\relax]
{\stepcounter{thehandwritten}
  \begin{frame}[fragile,plain,environment=handwritten]
    \ifthenelse{\equal{#1}{\relax}}{\def\list{22.5cm}}{\def\list{#1,22.5cm}}
    \foreach \vi [count=\i] in \list{
      \only<\i->{\insertpdfpage[\vi]{\handwrittensource}{\value{thehandwritten}}}
    }
}
{\end{frame}}

% \setmainfont{Latin Modern Sans}
\setmathfont{Latin Modern Math}
\setmathfont{Asana Math}[range={scr}]
\setmathfont{STIX Two Math-Regular}[range={bb}]
\setmathfont{Asana Math}[range={8709}]  % U+2205, emptyset
\setmathfont{Asana Math}[range={10631, 10632}]  % U+2987, U+2988, bb parenthesis

\usepackage{ulem}
\renewcommand{\ULdepth}{1.8pt}
\newcommand{\ul}[1]{\uline{#1}}

\newtheorem{izrek}{Izrek}
\newtheorem{trditev}{Trditev}
\newtheorem{lema}{Lema}
\newtheorem*{posledica}{Posledica}

\theoremstyle{definition}
\newtheorem{definicija}{Definicija}

\newtheorem*{primer}{Primer}
\newtheorem*{primer*}{Primer}
\newtheorem*{primeri}{Primeri}

\theoremstyle{remark}
\newtheorem*{opomba}{Opomba}

% \beamertemplatetransparentcovereddynamic

\title{Topological models}
\author{Luna Strah}
\institute{Univerza v Ljubljani, Fakulteta za matematiko in fiziko}
\date{19.~5.~2025}

% \newcommand{\vphi}{\phi}
\renewcommand{\phi}{\varphi}
\newcommand{\eps}{\varepsilon}
\renewcommand{\hat}{\widehat}
\renewcommand{\tilde}{\widetilde}
% \newcommand{\oldbar}{\bar}
% \renewcommand{\bar}{\overline}
\newcommand{\subs}{\subseteq}
\newcommand{\nin}{\not\in}

\newcommand{\p}[1]{\left( {#1} \right)}
\renewcommand{\b}[1]{\left[ {#1} \right]}
\newcommand{\set}[2]{\left\{ #1 \mid #2 \right\}}
\newcommand{\newfrac}[2]{{}^{#1}\!/_{\!#2}}
\newcommand{\smallfrac}[2]{{\textstyle\frac{#1}{#2}}}
\newcommand{\im}[1]{\mathrm{im}{\p{#1}}}
\newcommand{\mb}[1]{\mathbb{#1}}
\newcommand{\mf}[1]{\mathfrak{#1}}
\newcommand{\mc}[1]{\mathcal{#1}}
\newcommand{\id}{\mathrm{id}}

\def\bN{\mb{N}}
\def\bR{\mb{R}}
\def\subq{\subseteq}
\def\forces{\Vdash}
\def\phi{\varphi}

\makeatletter
\NewDocumentCommand{\@definstance}{mmm}{%
  \ExpandArgs{c}\NewDocumentCommand{#1}{s}{%
    \IfBooleanTF##1%
    {\textnormal{\sffamily #2}}%
    {\textnormal{\sffamily #3}}%
  }%
  \AtEndPreamble{%
    \pdfstringdefDisableCommands{%
      \expandafter\def\csname #1\endcsname*{#2}%
    }%
  }%
}
\newcommand{\definstance}[1]{\@definstance{#1}{\MakeUppercase{#1}}{\MakeLowercase{#1}}}
\makeatother
\newcommand{\instance}[1]{\textnormal{\sffamily #1}} % TODO: rename
\definstance{lem}
\definstance{lpo}
\definstance{alpo}
\definstance{aks}
\newcommand{\AC}{\instance{AC}}
\newcommand{\IAC}{\instance{IAC}}
\newcommand{\CC}{\instance{CC}}
\newcommand{\CCv}{\instance{CC}^∨}
\def\Rd{\bR_d}
\def\Rc{\bR_c}
\DeclareMathOperator{\dom}{dom}
%\DeclareMathOperator{\cov}{Cov}
\AtBeginDocument{
  %\def\c#1{\left( {#1} \right)^c}
  %\def\c#1{{#1}^c}
  \def\c{\uline}
  \renewcommand{\b}[1]{\left\{ {#1} \right\}}
  \renewcommand{\O}[1]{\mathcal{O}{#1}}
  \renewcommand{\int}{\textrm{Int}}
  \newcommand{\ext}{\textrm{Ext}}
}
\newcommand{\defquantifier}[2]{%
  \expandafter\newcommand\csname #1\endcsname[2]{{#2 ##1.}\;##2}%
}
\defquantifier{for}{\forall}
\defquantifier{exist}{\exists}
\defquantifier{unique}{\exists!}
\defquantifier{eventually}{\nabla}


%\renewcommand{\i}[1]{⟦ #1 ⟧}
% \newcommand{\brsem}[1]{⟬ #1 ⟭}
% \newcommand{\brsem}[1]{⦅ #1 ⦆}
\newcommand{\brsem}[1]{⦇ #1 ⦈}

\newcommand{\quot}[1]{``#1''}

% \setbeameroption{hide notes} % Only slides
% \setbeameroption{show only notes} % Only notes
% \setbeameroption{show notes} % Both
\setbeameroption{show notes on second screen=right} % Both

\usepackage{hyperref}
\hypersetup{pdfpagemode=FullScreen}

\def\handwrittensource{./topological-models.pdf}

% TODO: remove background and rules

\begin{document}
%%%%%
\begin{handwritten}
  % Title
  \note<1>{
    Today I will be giving an introduction to topological models in the sense of
    categories of Heyting valued sets.
    Some weeks ago prof. Simpson presented a tutorial on sheaf semantics, in
    much greater generality than what I will present today, and there you can
    just say \quot{topological models are categories of sheaves over a
      topological space} and be done.
    However in this case the objects involved are quite complicated.
    Conceptually, sheaves are simple, but any particular sheaf is going to be
    unwieldy. Take for example the sheaf of natural numbers. It is the sheaf of
    locally constant maps from open subsets to the naturals. But in the world of
    Heyting valued sets, you can just say the natural numbers object is the set
    \(ℕ\).
  }
  \note<2>{
    So my primary sources for this is Michael Fourman and Dana Scott's paper from
    the 70s, Sheaves and Logic, which introduced the concept and the 3rd volume
    of Francis Borceux's Handbook of Categorical Algebra, which has quite a few
    useful lemmas. A big difference to those though is that I develop as much of
    the theory in the internal language, which I believe works quite well.

    I hope to end up finishing the construction of the equivalence between
    Heying valued sets and sheaves and the construction of the real numbers
    object in a topos. % NOTE: will I actually though?

    But to start small, we wish to construct a topos, whose internal logic has
    as its truth values the open sets of our topological space.
  }
\end{handwritten}
%%%%%
\begin{handwritten}
  % Step 1
  \note{
    So first we can construct a propositional logic with infinitary conjunctions
    and disjuncitons, and that is basically just the structure of the topology
    as a complete Heyting algebra. Equivalently a frame, so I will call it that
    from now. % NOTE: will I?
  }
\end{handwritten}
\begin{handwritten}[7cm,11cm,13cm]
  % Slovar
  \note<1>{
    So first, we can interpret truth and falsity. Truth should obviously be the
    whole space. 
  }
\end{handwritten}
\begin{handwritten}
  % LEM iff discrete
  % NOTE: nekje pred tem rabim povedat, da delamo z okoliši/T₀
  \note{
  }
\end{handwritten}
%%%%%
\begin{handwritten}
  % Step 2
  \note{
    The idea is to define a set of generators for \(ℱ\).
  }
\end{handwritten}
\begin{handwritten}
  % Generators
  \note{
  }
\end{handwritten}
\begin{handwritten}
  % ℒ-sets
  \note{
  }
\end{handwritten}
\begin{handwritten}
  % ℒ-morphisms
  \note{
  }
\end{handwritten}
%%%%%
\begin{handwritten}
  % Step 3
  \note{
  }
\end{handwritten}
\begin{handwritten}
  % Slovar
  \note{
  }
\end{handwritten}
\begin{handwritten}
  % ℒ-set
  \note{
  }
\end{handwritten}
\begin{handwritten}
  % ℒ-morphisms
  \note{
  }
\end{handwritten}
\begin{handwritten}
  % id, ∘, R(f(a))
  \note{
  }
\end{handwritten}
\begin{handwritten}
  % Constructions
  \note{
  }
\end{handwritten}
\begin{handwritten}
  % funext
  \note{
  }
\end{handwritten}
\begin{handwritten}
  % mono inj
  % TODO: also recreate external proof
  \note{
  }
\end{handwritten}
\begin{handwritten}
  % epi surj
  \note{
  }
\end{handwritten}
\begin{frame}

  \begin{align*}
    ⟦c = g(b)⟧
    &= ⟦c = g(b)⟧∧⟦b = b⟧\\
    &= ⋁_{a ∈ A}⟦c = g(b)⟧∧⟦b=f(a)⟧\\
    &≤ ⋁_{a ∈ A, b' ∈ B}⟦c = g(b')⟧∧⟦b'=f(a)⟧∧⟦b=f(a)⟧\\
    &= ⋁_{a ∈ A}⟦c = g∘f(a)⟧∧⟦b=f(a)⟧\\
    &= ⋁_{a ∈ A}⟦c = h∘f(a)⟧∧⟦b=f(a)⟧\\
    &≤ ⋁_{a ∈ A, b' ∈ B}⟦c = h(b')⟧∧⟦b'=f(a)⟧∧⟦b=f(a)⟧\\
    &≤ ⋁_{b' ∈ B}⟦c = h(b')⟧∧⟦b'=b⟧\\
    &= ⟦c = h∘1_B(b')⟧\\
    &= ⟦c = h(b')⟧
  \end{align*}

\end{frame}
\begin{handwritten}
  % iso bij
  \note{
  }
\end{handwritten}
%%%%%
\begin{handwritten}
  % 
  \note{
  }
\end{handwritten}
%%%%%

\begin{frame}
  \frametitle{Logika odprtih množic}

  \begin{align*}
    &⊤&&     X\\
    &⊥&&     ∅\\
    \uncover<2->{
    &U ∧ V&& U ∩ V\\
    &U ∨ V&& U ∪ V\\
    }
    \uncover<3->{
    &¬U&&    \int{\p{Uᶜ}}\\
    }
    \uncover<4->{
    &U ⇒ V&& \int{\p{V ∪ Uᶜ}} = ⋃\set{W ⊆ X}{W ∩ U ⊆ V}\\
    &U ⇔ V&& \int{\p{\p{V∩U} ∪ \p{V∪U}ᶜ}}
    }
  \end{align*}
  
  \note{
    Zdej, prvo vprašanje je, kako zgledajo odprte množice kot resničnostne
    vrednosti. Očitno če je neki res povsod je res, tako da resnica bo cel
    prostor. Obratno, neresnica bo prazna množica, torej da nikjer ni res.
    
    Naprej, recimo, da imamo dve funkciji, ena je pozitivna na \(U\), druga
    je pozitivna na \(V\), pol sta obe pozitivni na preseku \(U ∩ V\).
    Tako da konjunkcija bo presek. Podobno je disjunkcija unija.
    
    Negacija je prva neočitna, ker ne mormo vzet samo komplementa, ampak lahko
    pa vzamemo notranjost komplementa, oziroma zunanjost množice.
    
    % TODO: implikacija
  }
\end{frame}

\begin{frame}
  \frametitle{Logika odprtih množic}

  \begin{align*}
    &⊤&&     X\\
    &⊥&&     ∅\\
    &U ∧ V&& U ∩ V\\
    &U ∨ V&& U ∪ V\\
    &¬U&&    \int{\p{Uᶜ}}\\
    &U ⇒ V&& \int{\p{V ∪ Uᶜ}} = ⋃\set{W ⊆ X}{W ∩ U ⊆ V}\\
    &U ⇔ V&& \int{\p{\p{V∩U} ∪ \p{V∪U}ᶜ}}
  \end{align*}

  \note{
    % TODO: tabla, LEM in DeMorgan?

    Izključeno tretjo možnost in DeMorganov zakon se da povedati zgolj z neko
    formulo o resničnostnih vrednostih, zato smo lahko naredili to
    karakterizacijo, ampak temu ni nujno vedno res. Obstajajo nekonstruktivni
    principi, ki govorijo recimo o neskončnih zaporedjih, ali pa realnih
    številih. Prav tako poznamo topološke lastnosti, ki govorijo o več kot le
    odprtih množicah, na primer \(T₆\) lastnost, ki pravi, da je vsaka zaprta
    množica natanko ničelna množica neke realne funkcije, in še druge. 
  }
\end{frame}

\begin{frame}
  \frametitle{Objekti}

  
  
  \note{
    Objekte v topoloških modelih se da konstruirat na veliko načinov, lahko so
    snopi, étale prostori, ali pa Heytingovo vrednotene množice. Jaz v delu
    uporabljam slednjo od teh, je pa bolj praktično rečt snopi. So pa te
    konstrukcije v vsakem primeru precej komplicirane, tako da se mi zdi da nima
    smisla, da katerokoli točno razpišem, tako da mi boste morali malo verjeti
    na besedo.

    Sicer je pa naša zgodba itak, da se stvari spreminjajo vzdolž topološkega
    prostora, tako da bi tudi želeli da se elementi spreminjajo vzdolž prostora.
    Tako da kar rečemo, da na vsaki točki prostora definiramo vrednost elementa,
    to je pa ubistvu kar funkcija iz prostora nekam (še ne vemo točno kam). Edino
    kar moramo paziti je, da je ta funkcija dovolj lepa (beri zvezna).
    In to dejansko večinoma dela, je pa kar dosti dela to dejansko preveriti,
    tko da ja, mi morte verjet :)
  }
\end{frame}

\begin{frame}
  \frametitle{Objekti}

  \(A\) množica, \(T\) topološki prostor
  \pause
  \[ T_X ≔ \set{f : U → T}{U ∈ 𝒪(X)} \]
  \pause
  \[ ℝ_X ≔ \set{f : U → ℝ}{U ∈ 𝒪(X)} = ⋃_{U ∈ 𝒪(X)}𝒞(U)\]
  Nad realnimi števili je torej \(\id : ℝ → ℝ\) realno število.
  \pause
  \[ \c A ≔ \set{f : U → A}{U ∈ 𝒪(X)} \]
  \[ \c ℕ ≔ \set{f : U → ℕ}{U ∈ 𝒪(X)} \]

  
  
  \note{
    Če malo fiksiramo oznake, naj bo …

    Najprej vložimo prostor \(T\), ker je še najbolj očitno kako se to naredi.
    Lahko bi vzeli kar zvezne funkcije iz \(X\) v \(T\). To bi delalo, ampak
    spomnimo se, da so naše resničnostne vrednosti odprte podmnožice \(X\).
    In obstoj elementa ima resničnostno vrednost, tako da je smiselno, da
    dovoljujemo tako imenovane delne elemente, torej elemente, ki niso
    definirani na celem \(X\). To pa pomeni, da je množica…

    Realna števila so potem kar realne funkcije, in recimo če je \(X = ℝ\) je
    identiteta neko realno število (reče se mu generični element).

    Za splošne množice pa vzamemo kar isto stvar. Ampak zdaj je vprašanje,
    kakšne funkcije vzamemo. Izkaže se da kar zvezne, kjer \(A\) opremimo z
    diskretno topologijo.
  }
\end{frame}

\begin{frame}
  \frametitle{Kvantifikatorji}

  \begin{align*}
    \uncover<2->{U ≤ }\for{y : Y}{P(y)} &≔
    \uncover<2->{U ≤ }⋀_{y ∈ Y} P(y)\\
    \uncover<2->{&⇔ \for{y ∈ Y}{\dom{y} ≤ U ⇒ \dom{y} ≤ P(y)}}\\
    % TODO: add pause
    \uncover<2->{U ≤ }\exist{y : Y}{P(y)} &≔
    \uncover<2->{U ≤ }⋁_{y ∈ Y} P(y)\\
    \uncover<2->{&⇔ \eventually{V ≤ U}{\exist{y ∈ Y}{\dom{y} = V ∧ \dom{y} ≤ P(y)}}}
  \end{align*}

  \note{

  }
\end{frame}

\begin{frame}
  \frametitle{Realna števila}

  \begin{trditev}
    Nad \(X\) drži \(\alpo*\) natanko tedaj, ko so ničelne množice funkcij
    \(U → ℝ\) odprte.
  \end{trditev}

  \pause
  
  \begin{trditev}
    Če je \(X\) (lokalno) \(T₆\), nad njem drži \(\aks*\).
  \end{trditev}

  \pause

  \begin{trditev}
    Če je prostor lokalno povezan in nad njem velja \(\Rd = \Rc\), velja tudi \(\alpo*\).
  \end{trditev}

  \note{
    \begin{align*}
      \alpo* &≔ \for{x : ℝ}{x > 0 ∨ x ≤ 0} = \for{x : ℝ}{x > 0 ∨ x = 0 ∨ x < 0}\\
      \aks*  &≔ \for{U : 𝒪(X)}{\exist{x : ℝ}{U ⇔ x > 0}}
    \end{align*}
  }
\end{frame}

\begin{frame}
  \frametitle{Ne}

  Verjetno \(\{0\}∪\set{2⁻ⁿ}{n ∈ ℕ}\) dela.

  \note{
    Tam sem se prepričala, a ne dokazala, da \(\alpo*\) in \(\CCv\) ne držita, pa
    vseeno \(\Rd = \Rc\).
  }
\end{frame}

\begin{frame}
  \frametitle{Izbira?}

  \begin{izrek}
    Nad \(T₁\) prostori velja števna izbira natanko tedaj, ko je topologija zaprta
    za števne preseke.
  \end{izrek}

  \pause
  
  \begin{izrek}
    Nad \(T₁\) prostori velja števna izbira natanko tedaj, ko velja \(\AC{\p{ℕ, 2}}\).
  \end{izrek}

  \pause

  \begin{trditev}[Hendtlass, Lubarsky 2016]
    Če je prostor ultraparakompakten, nad njem velja odvisna izbira.
  \end{trditev}

  \note{
    Tu števnost ni važna.
  }
\end{frame}

\begin{frame}
  \frametitle{}

  Vprašanja?

  \note{
  }
\end{frame}


\end{document}
