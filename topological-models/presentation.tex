\documentclass[handout]{beamer}
%\documentclass{beamer}

% \usetheme{boxes} % see http://www.deic.uab.es/~iblanes/beamer_gallery/ for lots of examples
\usetheme{metropolis}
\usecolortheme{rose}
% \useinnertheme{circles}
% \useoutertheme{split}
% \setbeamertemplate{blocks}[rounded][shadow=true]
\geometry{}

\setbeamertemplate{navigation symbols}{} % remove navigation symbols
\setbeamertemplate{footline}{} % remove title, too long

% %next set colors - not needed
% \setbeamercolor{title}{fg=black!70!black}
% \setbeamercolor{frametitle}{fg=blue!70!black}
% \setbeamercolor{framesubtitle}{fg=green!30!black}
% \setbeamercolor{author}{fg=red!70!black}
% \setbeamercolor{institute}{fg=green!30!black}
% \setbeamercolor{date}{fg=blue!50!black}

% \usepackage[T1]{fontenc}        % kodiranje znakov v .pdf
% \usepackage[utf8]{inputenc}     % kodiranje znakov v .tex
% \usepackage[slovene]{babel}     % nastavimo slovenščino
% \usepackage{stmaryrd}

\usepackage{fontspec}
\usepackage{unicode-math}
\usepackage{enumerate}

\usepackage{graphicx}
\usepackage{ifthen}

\newcounter{thehandwritten}
\newcommand{\insertpdfpage}[3][22.5cm]{
  \begin{tikzpicture}[remember picture,overlay]
    \node[at=(current page.north),below] {
      \includegraphics[keepaspectratio=true,width=\paperwidth,page=#3,trim={0 {\dimexpr 22.5cm-#1\relax} 0 0},clip]{#2}
    };
  \end{tikzpicture}
}
\newenvironment{handwritten}[1][\relax]
{\stepcounter{thehandwritten}
  \begin{frame}[fragile,plain,environment=handwritten]
    \ifthenelse{\equal{#1}{\relax}}{\def\list{22.5cm}}{\def\list{#1,22.5cm}}
    \foreach \vi [count=\i] in \list{
      \only<\i->{\insertpdfpage[\vi]{\handwrittensource}{\value{thehandwritten}}}
    }
}
{\end{frame}}

% \setmainfont{Latin Modern Sans}
\setmathfont{Latin Modern Math}
\setmathfont{Asana Math}[range={scr}]
\setmathfont{STIX Two Math-Regular}[range={bb}]
\setmathfont{Asana Math}[range={8709}]  % U+2205, emptyset
\setmathfont{Asana Math}[range={10631, 10632}]  % U+2987, U+2988, bb parenthesis

\usepackage{ulem}
\renewcommand{\ULdepth}{1.8pt}
\newcommand{\ul}[1]{\uline{#1}}

\newtheorem{izrek}{Izrek}
\newtheorem{trditev}{Trditev}
\newtheorem{lema}{Lema}
\newtheorem*{posledica}{Posledica}

\theoremstyle{definition}
\newtheorem{definicija}{Definicija}

\newtheorem*{primer}{Primer}
\newtheorem*{primer*}{Primer}
\newtheorem*{primeri}{Primeri}

\theoremstyle{remark}
\newtheorem*{opomba}{Opomba}

% \beamertemplatetransparentcovereddynamic

\title{Topological models}
\author{Luna Strah}
\institute{Univerza v Ljubljani, Fakulteta za matematiko in fiziko}
\date{19.~5.~2025}

% \newcommand{\vphi}{\phi}
\renewcommand{\phi}{\varphi}
\newcommand{\eps}{\varepsilon}
\renewcommand{\hat}{\widehat}
\renewcommand{\tilde}{\widetilde}
% \newcommand{\oldbar}{\bar}
% \renewcommand{\bar}{\overline}
\newcommand{\subs}{\subseteq}
\newcommand{\nin}{\not\in}

\newcommand{\p}[1]{\left( {#1} \right)}
\renewcommand{\b}[1]{\left[ {#1} \right]}
\newcommand{\set}[2]{\left\{ #1 \mid #2 \right\}}
\newcommand{\newfrac}[2]{{}^{#1}\!/_{\!#2}}
\newcommand{\smallfrac}[2]{{\textstyle\frac{#1}{#2}}}
\newcommand{\im}[1]{\mathrm{im}{\p{#1}}}
\newcommand{\mb}[1]{\mathbb{#1}}
\newcommand{\mf}[1]{\mathfrak{#1}}
\newcommand{\mc}[1]{\mathcal{#1}}
\newcommand{\id}{\mathrm{id}}

\def\bN{\mb{N}}
\def\bR{\mb{R}}
\def\subq{\subseteq}
\def\forces{\Vdash}
\def\phi{\varphi}

\makeatletter
\NewDocumentCommand{\@definstance}{mmm}{%
  \ExpandArgs{c}\NewDocumentCommand{#1}{s}{%
    \IfBooleanTF##1%
    {\textnormal{\sffamily #2}}%
    {\textnormal{\sffamily #3}}%
  }%
  \AtEndPreamble{%
    \pdfstringdefDisableCommands{%
      \expandafter\def\csname #1\endcsname*{#2}%
    }%
  }%
}
\newcommand{\definstance}[1]{\@definstance{#1}{\MakeUppercase{#1}}{\MakeLowercase{#1}}}
\makeatother
\newcommand{\instance}[1]{\textnormal{\sffamily #1}} % TODO: rename
\definstance{lem}
\definstance{lpo}
\definstance{alpo}
\definstance{aks}
\newcommand{\AC}{\instance{AC}}
\newcommand{\IAC}{\instance{IAC}}
\newcommand{\CC}{\instance{CC}}
\newcommand{\CCv}{\instance{CC}^∨}
\def\Rd{\bR_d}
\def\Rc{\bR_c}
\DeclareMathOperator{\dom}{dom}
%\DeclareMathOperator{\cov}{Cov}
\AtBeginDocument{
  %\def\c#1{\left( {#1} \right)^c}
  %\def\c#1{{#1}^c}
  \def\c{\uline}
  \renewcommand{\b}[1]{\left\{ {#1} \right\}}
  \renewcommand{\O}[1]{\mathcal{O}{#1}}
  \renewcommand{\int}{\textrm{Int}}
  \newcommand{\ext}{\textrm{Ext}}
}
\newcommand{\defquantifier}[2]{%
  \expandafter\newcommand\csname #1\endcsname[2]{{#2 ##1.}\;##2}%
}
\defquantifier{for}{\forall}
\defquantifier{exist}{\exists}
\defquantifier{unique}{\exists!}
\defquantifier{eventually}{\nabla}


%\renewcommand{\i}[1]{⟦ #1 ⟧}
% \newcommand{\brsem}[1]{⟬ #1 ⟭}
% \newcommand{\brsem}[1]{⦅ #1 ⦆}
\newcommand{\brsem}[1]{⦇ #1 ⦈}

\newcommand{\quot}[1]{``#1''}

% \setbeameroption{hide notes} % Only slides
% \setbeameroption{show only notes} % Only notes
% \setbeameroption{show notes} % Both
\setbeameroption{show notes on second screen=right} % Both

\usepackage{hyperref}
\hypersetup{pdfpagemode=FullScreen}

\def\handwrittensource{./topological-models.pdf}

% TODO: remove background and rules

\begin{document}
%%%%%
\begin{handwritten}
  % Title
  \note<1|handout:1>{
    Today I will be giving an introduction to topological models in the sense of
    categories of Heyting valued sets.
    Some weeks/months ago prof. Simpson presented a tutorial on sheaf semantics,
    in much greater generality than what I will present today, and there you can
    just say \quot{topological models are categories of sheaves over a
      topological space} and be done.
    However in this case the objects involved are quite complicated.
    Conceptually, sheaves are simple, but any particular sheaf is going to be
    unwieldy. Take for example the sheaf of natural numbers. It is the sheaf of
    locally constant maps from open subsets to the naturals. But in the world of
    Heyting valued sets, you can just say the natural numbers object is the set
    \(ℕ\).
  }
  \note<2|handout:2>{
    So my primary sources for this is Michael Fourman and Dana Scott's paper from
    the 70s, Sheaves and Logic, which introduced the concept and the 3rd volume
    of Francis Borceux's Handbook of Categorical Algebra, which has quite a few
    useful lemmas. A big difference to those though is that I develop as much of
    the theory in the internal language, which I believe works quite well.

    I hope to end up finishing the construction of the equivalence between
    Heying valued sets and sheaves and the construction of the real numbers
    object in a topos. % NOTE: will I actually though?

    But to start small, we wish to construct a topos, whose internal logic has
    as its truth values the open sets of our topological space.

  }
\end{handwritten}
%%%%%
\begin{handwritten}
  % Step 1
  \note{
    So first we can construct a propositional logic with infinitary conjunctions
    and disjuncitons, and that is basically just the structure of the topology
    as a complete Heyting algebra. Equivalently a frame, so I will call it that
    from now. % NOTE: will I?
  }
\end{handwritten}
\begin{handwritten}[7cm,11cm,13cm]
  % Slovar
  \note<1>{
    So first, we can interpret truth and falsity. Truth should obviously be the
    whole space.
    % TODO: the rest of the owl
  }
\end{handwritten}
\begin{handwritten}[6cm]
  % LEM iff discrete
  % NOTE: nekje pred tem rabim povedat, da delamo z okoliši/T₀
  \note{
    Now, to demonstrate how expressive this logic is, we can actually prove the
    following.
    
  }
\end{handwritten}
%%%%%
\begin{handwritten}
  % Step 2
  \note{
    As I mentioned earlier the idea goes back to Peter Fourman and Dana Scott's
    work in the 70s, but they didn't explicitly write what the motivation for
    this construction is, at least not one that immediately stands out to me.
    To be precise, they say this is \quot{the obvious extension of Heyting
      algebras to predicate logic}.

    This is also why I also rely heavily on Borceux's textbook, which
    \emph{does} explicitly provide a motivation. In algebra, it is often useful
    to give an algebraic structure via its presentation, aka a set of generators
    and some relations they need to satisfy. We wish to apply the same idea to
    sheaves, and it turns out it works quite well.
    We know sheaves are closed under restrictions and gluing, and those are the
    only forms of generation that sheaves support. Furthermore restrictions
    distribute over gluing, so this gives a somewhat natural characterisation of
    generation.

  }
\end{handwritten}
\begin{handwritten}[12cm]
  % Generators
  % TODO: change F' to S
  \note<1>{
    So in particular, how to define that a set generates a sheaf, we first say
    it's a subset of it's set of elements, and it generates \(ℱ\) when \(ℱ\) is
    the least subsheaf of \(ℱ\) that contains \(S\).
  }
  \note<2>{
    Equivalently, this is a set of generators as discussed above. Key to note
    here is, that restriction is a bona fide operation, so we don't need any
    equivalence relation. So if we wish to consider generating sets in absence
    of the sheaf we are generating, we also need to define it. And it turns out
    a set of elements and an appropriate notion of equivalence relation is
    exactly what we need to recover a whole sheaf.
  }
\end{handwritten}
\begin{handwritten}
  % ℒ-sets
  % TODO: add definition of ‖a‖
  \note<1|handout:1>{
    Let \(ℒ\) be the topology on \(X\). Here \(ℒ\) stands for \emph{locale}, but
    I don't say this in my thesis.
    So an \(ℒ\)-set is just a set, with a (partial) equivalence relation in the
    internal language, which we call the \(ℒ\)-equality. But because we don't
    have an internal language yet, we have to say it's a map from \(A×A\) to
    \(ℒ\), which satisfies the symmetry-like condition and the transitivity-like
    condition. We ommit reflexivity in a sense, since our elements are not
    necessarily total, e.g. \(\frac1x\) is not defined at \(0\). However we will
    have that reflexivity holds up to where the element is defined, and in fact,
    we define the extent of definition by the diagonal of the \(ℒ\)-equality.
    
    I'm not quite set on the name, alternatively
    this is called a \quot{Heyting valued set}, but that's somewhat of a
    mouthful in English and a lot of a mouthful in Slovene, so for now I'm
    sticking with the short \(ℒ\)-set.
  }
  \note<2|handout:2>{
    The next building block in our story would be morphisms, and we might be
    tempted to just define them as structure preserving maps, so maps
    \(f : A → B\) such that \(a = a' ⇒ f(a) = f(a')\) and \(‖f(a)‖ = ‖a‖\).
    Why exactly these maps? Well, because they will work. But why am I writing
    this on the whiteboard? Because they don't work yet.

    Consider this example of a map we would be able to define internally.
  }
\end{handwritten}
\begin{handwritten}
  % Non-example
  \note{
    Let \(A\) be the \(ℒ\)-set with one global element \(a\) and \(B\) the
    \(ℒ\)-set with two elements \(b₀\) and \(b₁\), defined at \(0\) and \(1\)
    respectivelly.
    Then there should actually be only one map from \(A\) to \(B\). This is
    because \(b₀\) and \(b₁\) should actually glue to a single element \(b\) at
    the top, and then both \(A\) and \(B\) are one-element sets, so we only have
    one map.

    But we can define it sort of internally. At \(0\) the map maps \(a\) to
    \(b₀\) and at \(1\) it maps \(a\) to \(b₁\). So in essence, we can define a
    relation between \(A\) and \(B\), with the meaning of \quot{\(a\) is mapped
      to \(b\) by \(f\)}. And it will satisfy the internal version of functionality.
  }
\end{handwritten}
\begin{handwritten}
  % ℒ-morphisms
  \note{
    So an \(ℒ\)-morphism will be a functional relation in the internal language,
    and again that means we have a map from \(B×A\) to \(ℒ\) that satisfies the
    appropriate axioms, so that's \(3\) for single-valuedness and \(4\) for
    totality. And the other two are just saying that this relation
    sufficiently agrees with the \(ℒ\)-equality. And in fact, this is exactly
    what it means for a map like that to be a relation. Those are called
    strictness and extensionality, and will be essentialy tautologies in the
    internal language. But lets actually define the internal language first, so
    I stop gesturing to the air.
  }
\end{handwritten}
%%%%%
\begin{handwritten}
  % Step 3
  \note{
    And considering I already cheated a bit and gave myself infinitary
    conjunctions and disjunctions there's not much to add.
    The basic connectives are the same, and the quantifiers are essentially the
    same, except we account for the extents of elements.
    We can also interpret relations and equality as their \(ℒ\)-set
    counterparts.
  }
\end{handwritten}
\begin{handwritten}
  % Slovar
  % TODO: add relation?
  \note{
    So what changes is that for universal quantification we add that \(φ(a)\) must
    at least where \(a\) is defined, and for existential quantification \(a\)
    exists at most where \(a\) exists.

    We do this because sometimes we don't use all bound variables, e.g. in
    \(\exist{x : A}{⊤}\), if \(A\) only has one element defined at the open
    point of the Sierpinski space.
    Similarly, for universal quantification this allows us to say for example
    that \(\for{x : A}{x = x}\) holds.

    There is also a term language, but it's not that interesting, elements of
    \(ℒ\)-sets are interpreted as themselves, and internal operations are just
    external maps (that are strict and extensional).
    
    
    It's important to note though, that we can't actually form the term \(f(a)\)
    for an \(ℒ\)-morphism. We need an actual (extensional) mapping for that,
    though in certain contexts we will be able to use \(f(a)\) as a bona fide
    element of the codomain.
  }
\end{handwritten}
\begin{handwritten}
  % ℒ-set
  \note{
    So rephrasing the conditions in the definitions in terms of the internal
    language we get the following, which is saying \emph{exactly} \quot{a set
      with an internal equivalence relation}.
  }
\end{handwritten}
\begin{handwritten}
  % ℒ-morphisms
  \note{
    And similarly here we get \quot{functional relation} with \(3\) and \(4\),
    and the first two are essentially tautologies. For the first, to have formed
    \quot{\(b = f(a)\)} we needed to have \(b\) and \(a\) exist. But then the
    consequent is just true, so the first condition doesn't say anything. And in
    the second one, if we have equalities we can substitute with them, so it's a
    tautology again.
    % NOTE: Here it's somewhat nuanced whether we can actually substitute. It's
    % probably a consequence of the definition of relation.
  }
\end{handwritten}
\begin{handwritten}
  % Step 4
  \note{
    Now we're ready to explore the landscape of topological models in this
    context, from the internal view.
  }
\end{handwritten}
\begin{handwritten}[10cm]
  % id, ∘, R(f(a))
  \note{
    Obviously we need \(ℒ\)-sets to form a category, so we need an identity
    morphism and composition. But since morphisms are internal relations, we can
    just take whatever the identity and composition are for relations. In our
    case, the identity is just the \(ℒ\)-equality, and the composition is just
    what it is.

    And as promised, we can evaluate formulas at \(f(a)\), since propositionally
    there exists a \(b\) that is equal to \(f(a)\).
  }
\end{handwritten}
\begin{handwritten}
  % Constructions
  \note{
    I could have also done this earlier, but we can construct the usual
    suspects.
  }
\end{handwritten}
\begin{handwritten}
  % funext
  \note{
  }
\end{handwritten}
\begin{handwritten}
  % mono inj
  \note{
  }
\end{handwritten}
\begin{handwritten}
  % epi surj
  \note{
  }
\end{handwritten}
\begin{frame}

  \begin{align*}
    ⟦c = g(b)⟧
    &= ⟦c = g(b)⟧∧⟦b = b⟧\\
    &= ⋁_{a ∈ A}⟦c = g(b)⟧∧⟦b=f(a)⟧\\
    &≤ ⋁_{a ∈ A, b' ∈ B}⟦c = g(b')⟧∧⟦b'=f(a)⟧∧⟦b=f(a)⟧\\
    &= ⋁_{a ∈ A}⟦c = g∘f(a)⟧∧⟦b=f(a)⟧\\
    &= ⋁_{a ∈ A}⟦c = h∘f(a)⟧∧⟦b=f(a)⟧\\
    &= ⋁_{a ∈ A, b' ∈ B}⟦c = h(b')⟧∧⟦b'=f(a)⟧∧⟦b=f(a)⟧\\
    &≤ ⋁_{b' ∈ B}⟦c = h(b')⟧∧⟦b'=b⟧\\
    &= ⟦c = h∘1_B(b)⟧\\
    &= ⟦c = h(b)⟧
  \end{align*}

\end{frame}
\begin{handwritten}
  % iso bij
  \note{
  }
\end{handwritten}
%%%%%
\begin{handwritten}
  % 
  \note{
  }
\end{handwritten}
%%%%%
\begin{frame}
  \frametitle{Objekti}

  
  
  \note{
    Objekte v topoloških modelih se da konstruirat na veliko načinov, lahko so
    snopi, étale prostori, ali pa Heytingovo vrednotene množice. Jaz v delu
    uporabljam slednjo od teh, je pa bolj praktično rečt snopi. So pa te
    konstrukcije v vsakem primeru precej komplicirane, tako da se mi zdi da nima
    smisla, da katerokoli točno razpišem, tako da mi boste morali malo verjeti
    na besedo.

    Sicer je pa naša zgodba itak, da se stvari spreminjajo vzdolž topološkega
    prostora, tako da bi tudi želeli da se elementi spreminjajo vzdolž prostora.
    Tako da kar rečemo, da na vsaki točki prostora definiramo vrednost elementa,
    to je pa ubistvu kar funkcija iz prostora nekam (še ne vemo točno kam). Edino
    kar moramo paziti je, da je ta funkcija dovolj lepa (beri zvezna).
    In to dejansko večinoma dela, je pa kar dosti dela to dejansko preveriti,
    tko da ja, mi morte verjet :)
  }
\end{frame}

\begin{frame}
  \frametitle{Objekti}

  \(A\) množica, \(T\) topološki prostor
  \pause
  \[ T_X ≔ \set{f : U → T}{U ∈ 𝒪(X)} \]
  \pause
  \[ ℝ_X ≔ \set{f : U → ℝ}{U ∈ 𝒪(X)} = ⋃_{U ∈ 𝒪(X)}𝒞(U)\]
  Nad realnimi števili je torej \(\id : ℝ → ℝ\) realno število.
  \pause
  \[ \c A ≔ \set{f : U → A}{U ∈ 𝒪(X)} \]
  \[ \c ℕ ≔ \set{f : U → ℕ}{U ∈ 𝒪(X)} \]

  
  
  \note{
    Če malo fiksiramo oznake, naj bo …

    Najprej vložimo prostor \(T\), ker je še najbolj očitno kako se to naredi.
    Lahko bi vzeli kar zvezne funkcije iz \(X\) v \(T\). To bi delalo, ampak
    spomnimo se, da so naše resničnostne vrednosti odprte podmnožice \(X\).
    In obstoj elementa ima resničnostno vrednost, tako da je smiselno, da
    dovoljujemo tako imenovane delne elemente, torej elemente, ki niso
    definirani na celem \(X\). To pa pomeni, da je množica…

    Realna števila so potem kar realne funkcije, in recimo če je \(X = ℝ\) je
    identiteta neko realno število (reče se mu generični element).

    Za splošne množice pa vzamemo kar isto stvar. Ampak zdaj je vprašanje,
    kakšne funkcije vzamemo. Izkaže se da kar zvezne, kjer \(A\) opremimo z
    diskretno topologijo.
  }
\end{frame}

\begin{frame}
  \frametitle{Kvantifikatorji}

  \begin{align*}
    \uncover<2->{U ≤ }\for{y : Y}{P(y)} &≔
    \uncover<2->{U ≤ }⋀_{y ∈ Y} P(y)\\
    \uncover<2->{&⇔ \for{y ∈ Y}{\dom{y} ≤ U ⇒ \dom{y} ≤ P(y)}}\\
    % TODO: add pause
    \uncover<2->{U ≤ }\exist{y : Y}{P(y)} &≔
    \uncover<2->{U ≤ }⋁_{y ∈ Y} P(y)\\
    \uncover<2->{&⇔ \eventually{V ≤ U}{\exist{y ∈ Y}{\dom{y} = V ∧ \dom{y} ≤ P(y)}}}
  \end{align*}

  \note{

  }
\end{frame}

\begin{frame}
  \frametitle{Realna števila}

  \begin{trditev}
    Nad \(X\) drži \(\alpo*\) natanko tedaj, ko so ničelne množice funkcij
    \(U → ℝ\) odprte.
  \end{trditev}

  \pause
  
  \begin{trditev}
    Če je \(X\) (lokalno) \(T₆\), nad njem drži \(\aks*\).
  \end{trditev}

  \pause

  \begin{trditev}
    Če je prostor lokalno povezan in nad njem velja \(\Rd = \Rc\), velja tudi \(\alpo*\).
  \end{trditev}

  \note{
    \begin{align*}
      \alpo* &≔ \for{x : ℝ}{x > 0 ∨ x ≤ 0} = \for{x : ℝ}{x > 0 ∨ x = 0 ∨ x < 0}\\
      \aks*  &≔ \for{U : 𝒪(X)}{\exist{x : ℝ}{U ⇔ x > 0}}
    \end{align*}
  }
\end{frame}

\begin{frame}
  \frametitle{Ne}

  Verjetno \(\{0\}∪\set{2⁻ⁿ}{n ∈ ℕ}\) dela.

  \note{
    Tam sem se prepričala, a ne dokazala, da \(\alpo*\) in \(\CCv\) ne držita, pa
    vseeno \(\Rd = \Rc\).
  }
\end{frame}

\begin{frame}
  \frametitle{Izbira?}

  \begin{izrek}
    Nad \(T₁\) prostori velja števna izbira natanko tedaj, ko je topologija zaprta
    za števne preseke.
  \end{izrek}

  \pause
  
  \begin{izrek}
    Nad \(T₁\) prostori velja števna izbira natanko tedaj, ko velja \(\AC{\p{ℕ, 2}}\).
  \end{izrek}

  \pause

  \begin{trditev}[Hendtlass, Lubarsky 2016]
    Če je prostor ultraparakompakten, nad njem velja odvisna izbira.
  \end{trditev}

  \note{
    Tu števnost ni važna.
  }
\end{frame}

\begin{frame}
  \frametitle{}

  Vprašanja?

  \note{
  }
\end{frame}


\end{document}
